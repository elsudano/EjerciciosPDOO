%%%%%%%%%%%%%%%%%%%%%
% Short Sectioned Assignment LaTeX Template Version 1.0 (5/5/12)
% This template has been downloaded from: http://www.LaTeXTemplates.com
% Original author:  Frits Wenneker (http://www.howtotex.com)
% License: CC BY-NC-SA 3.0 (http://creativecommons.org/licenses/by-nc-sa/3.0/)
%%%%%%%%%%%%%%%%%%%%%

%----------------------------------------------------------------------------------------
%	PACKAGES AND OTHER DOCUMENT CONFIGURATIONS
%----------------------------------------------------------------------------------------

\documentclass[paper=a4, fontsize=10pt]{scrartcl} % A4 paper and 11pt font size

% ---- Margenes y Tamaño del Folio -----
\usepackage{geometry}
\geometry{a4paper, total={210mm,297mm}, left=25mm, right=25mm, top=20mm, bottom=25mm,}

% ---- Entrada y salida de texto -----
\usepackage[T1]{fontenc} % Use 8-bit encoding that has 256 glyphs
\usepackage[utf8]{inputenc}
%\usepackage{fourier} % Use the Adobe Utopia font for the document - comment this line to return to the LaTeX default


% ---- Idioma y Figuras --------
% Selecciona el español para palabras introducidas automáticamente, p.ej. "septiembre" en la fecha
% y especifica que se use la palabra Tabla en vez de Cuadro
\usepackage[spanish, es-tabla]{babel} 
\usepackage{wrapfig} % es para que las imagenes puedan estar entre el texto
\usepackage{subcaption} % para las imagenes en columnas
%\usepackage{svg} % paquete para poder insertar graficos en formato svg y que se impriman en pdf

% ---- Otros paquetes ----
\usepackage{amsmath,amsfonts,amsthm} % Math packages
\usepackage{graphics,graphicx,floatrow} %para incluir imágenes y notas en las imágenes
\usepackage{lscape, pdflscape} %Permite poner paginas "tumbadas" en un documento con las paginas "rectas".
\usepackage{pgfgantt} % Para realizar los diagramas de Gantt
\usepackage{fancyhdr, extramarks} % Custom headers and footers

% ---------- Para hacer tablas comlejas ---------
\usepackage{multirow,threeparttable}
\usepackage{multicol} % Required for creating multiple columns in slides
\usepackage{sectsty} % Allows customizing section commands
\usepackage{tabularx} % Tables with width auto ajust
%\usepackage{tablefootnote} % para poner pie de notas en las tablas solo en tabular envirotment

%------------------------------------------------
%                  Links y Moneda
\usepackage[hidelinks]{hyperref} % para los enlaces de la bibliografía
\hypersetup{
	colorlinks,
	citecolor=blue,
	filecolor=brown,
	linkcolor=black,
	urlcolor=blue
}
\usepackage{eurosym} % para el euro

%------------------------------------------------
% Colors
%\usepackage[usenames,table]{xcolor}	 % Required for custom colors only with tikz package
\usepackage{xcolor, colortbl} % color en las tablas
% Define a few colors for making text stand out within the presentation
\definecolor{mygreen}{RGB}{44,85,17}
\definecolor{myblue}{RGB}{34,31,217}
\definecolor{mybrown}{RGB}{194,164,113}
\definecolor{myred}{RGB}{255,66,56}
\definecolor{mygrey}{RGB}{230,230,230}
% Use these colors within the presentation by enclosing text in the commands below
\newcommand*{\mygreen}[1]{\textcolor{mygreen}{#1}}
\newcommand*{\myblue}[1]{\textcolor{myblue}{#1}}
\newcommand*{\mybrown}[1]{\textcolor{mybrown}{#1}}
\newcommand*{\myred}[1]{\textcolor{myred}{#1}}
\newcommand*{\mygrey}[1]{\textcolor{mygrey}{#1}}

%------------------------------------------------
%             Listas
\usepackage{listings} % para los listados de código
\usepackage{tabto} % para utilizar los tab en las listas
\lstdefinestyle{C++}{
	belowcaptionskip=1\baselineskip,
	frame=L,
	xleftmargin=\parindent,
	showstringspaces=false,
	basicstyle=\footnotesize\ttfamily,
	keywordstyle=\bfseries\color{green!40!black},
	%commentstyle=\itshape\color{purple!40!black},
	commentstyle=\color{gray}\upshape,
	identifierstyle=\color{blue},
	stringstyle=\color{orange},
	%tagstyle=\color{darkblue}\bf,
	language=C++,
	extendedchars=true, 
	breaklines=true,
	breakatwhitespace=true,
	emph={},
	emphstyle=\color{red},
	columns=fullflexible,
	morestring=[b]",
	morecomment=[s]{/**}{*/},
	morecomment=[s][\color{forestgreen}]{<!--}{-->},
	otherkeywords={},
	morekeywords={IP,num,string,Servicio}
}
\lstdefinestyle{java}{
	language=Java,
	aboveskip=3mm,
	belowskip=3mm,
	showstringspaces=false,
	columns=flexible,
	basicstyle={\footnotesize\ttfamily},
	numberstyle={\tiny},
	numbers=left,
	keywordstyle=\color{blue},
	commentstyle=\color{dkgreen},
	stringstyle=\color{mauve},
	breaklines=true,
	breakatwhitespace=true,
	tabsize=3
}

\lstdefinestyle{bytecode}{
	otherkeywords={invokedynamic},
	language=JVMIS,
	aboveskip=3mm,
	belowskip=3mm,
	showstringspaces=false,
	columns=flexible,
	basicstyle={\footnotesize\ttfamily},
	numberstyle={\tiny},
	numbers=left,
	keywordstyle=\color{blue},
	commentstyle=\color{dkgreen},
	stringstyle=\color{mauve},
	breaklines=true,
	breakatwhitespace=true,
	tabsize=3
}

\lstdefinestyle{fsharp} {	
	otherkeywords={ let!, return!, do!, yield!, use!, var, from, select, where, order},
	keywordstyle=\color{blue},
	sensitive=true,
	aboveskip=3mm,
	belowskip=3mm,
	showstringspaces=false, 
	columns=flexible,
	basicstyle={\footnotesize\ttfamily},
	numberstyle={\tiny},
	numbers=left,
	breaklines=true,
	upquote=true,
	tabsize=3,
	morecomment=[l][\color{dkgreen}]{///},
	morecomment=[l][\color{dkgreen}]{//},
	morecomment=[s][\color{dkgreen}]{{(*}{*)}},
	morestring=[b]",
	showstringspaces=false,
	literate={`}{\`}1,
	stringstyle=\color{mauve},
	morekeywords={let, new, match, with, rec, 
		open, module, namespace, type, of, member, 
		and, for, while, true, false, in, do, begin, 
		end, fun, function, return, yield, try, val, 
		mutable, if, then, else, cloud, async, static, 
		use, abstract, interface, inherit, finally, maybe, option }
}

\lstdefinestyle{csharp} {
	aboveskip=3mm,
	belowskip=3mm,
	showstringspaces=false,
	columns=flexible,
	basicstyle={\footnotesize\ttfamily},
	numberstyle={\tiny},
	numbers=left,
	keywordstyle=\color{blue},
	commentstyle=\color{dkgreen},
	stringstyle=\color{mauve},
	breaklines=true,
	breakatwhitespace=true,
	tabsize=3,
	morecomment = [l]{//}, 
	morecomment = [l]{///},
	morecomment = [s]{/*}{*/},
	morestring=[b]", 
	sensitive = true,
	morekeywords = {async, await, abstract,  
		event,  new,  struct,
		as,  explicit,  null,  switch,
		base,  extern,  object,  this,
		bool,  false,  operator,  throw,
		break,  finally,  out,  true,
		byte,  fixed,  override,  try,
		case,  float,  params,  typeof,
		catch,  for,  private,  uint,
		char,  foreach,  protected,  ulong,
		checked,  goto,  public,  unchecked,
		class,  if,  readonly,  unsafe,
		const,  implicit,  ref,  ushort,
		continue,  in,  return,  using,
		decimal,  int,  sbyte,  virtual,
		default,  interface,  sealed,  volatile,
		delegate,  internal,  short,  void,
		do,  is,  sizeof,  while,
		double,  lock,  stackalloc,   
		else,  long,  static,   
		enum,  namespace,  string }
}

\lstdefinestyle{scala} {  
	morekeywords={ abstract,case,catch,
		char,class,
		def,else,extends,final,
		if,import,
		match,module,new,null,object,
		override,package,private,protected,
		public,return,super,this,throw,
		trait,try,type,val,var,with,implicit,
		macro,sealed
	},
	sensitive,
	morecomment=[l]//,
	morecomment=[s]{/*}{*/},
	morestring=[b]",
	morestring=[b]',
	aboveskip=3mm,
	belowskip=3mm,
	showstringspaces=false,
	columns=flexible,
	basicstyle={\footnotesize\ttfamily},
	numberstyle={\tiny},
	numbers=left,
	keywordstyle=\color{blue},
	commentstyle=\color{dkgreen},
	stringstyle=\color{mauve},
	breaklines=true,
	breakatwhitespace=true,
	tabsize=3
}
\lstset{ %
	literate={á}{{\'a}}1 {ó}{{\'o}}1 {ñ}{{\~n}}1 {é}{{\'e}}1 {í}{{\'i}}1 {ú}{{\'u}}1, % esto es para los acentos en utf8
	backgroundcolor=\color{mygrey},     % elegir el color de fondo; debe agregar \usepackage{color} o \usepackage{xcolor}
	basicstyle=\footnotesize\ttfamily,                 % el tamaño de las fuentes que se utilizan para el código
	breakatwhitespace=false,                % establece si se rompe automáticos sólo debe ocurrir en los espacios en blanco
	breaklines=true,                               % establece los saltos de línea automática
	captionpos=b,                                  % establece el título de la posición hacia abajo
	commentstyle=\color{magenta},         % estilo de comentarios
	deletekeywords={...},                      % si desea eliminar palabras clave de la lengua dada
	escapeinside={\%*}{*)},                % si desea agregar LaTeX dentro de su código
	extendedchars=true,                        % le permite utilizar caracteres no ASCII; de 8-bits de codificaciones sólo, no funciona con UTF-8
	frame=single,                                   % agrega un marco alrededor del código
	keepspaces=true,                            % mantiene espacios de texto, útiles para mantener la sangría de código (posiblemente necesita columnas = flexible)
	keywordstyle=\color{blue},              % estilo palabra clave
	%language=bash,                            % el lenguaje del código
	morekeywords={*,...},                     % si quieres añadir más palabras clave para el conjunto
	numbers=left,                                   % dónde poner la línea de números; los valores posibles son (ninguno, izquierda, derecha)
	numbersep=5pt,                               % hasta qué punto las alineaciones números son a partir del código
	numberstyle=\footnotesize\color{blue},        % el estilo que se utiliza para los números de línea-
	rulecolor=\color{black},                    % si no se establece, el marco-color puede ser cambiado en saltos de línea en el texto no-negro (por ejemplo, los comentarios (verde aquí))
	showspaces=false,                            % mostrar espacios añadiendo todas partes subrayado particulares; anula 'showstringspaces'
	showstringspaces=false,                   % subrayar espacios dentro de las cadenas sólo
	showtabs=false,                                % Mostrar pestañas dentro de las cadenas de subrayado particulares
	stepnumber=1,                                 % el paso entre dos números de línea. Si es 1, se numerará cada línea
	stringstyle=\color{red},              % cadena estilo literal
	tabsize=2,                                         % establece TABSIZE predeterminado a 2 espacios
	title=\lstname                                   % mostrar el nombre de archivo de los archivos incluidos con \ lstinputlisting; también tratar de leyenda en vez de título
}
%------------------------------------------------

\pagestyle{fancyplain} % Makes all pages in the document conform to the custom headers and footers
\fancyhead[LO]{Solución a Problemas de Abstracción} % No page header - if you want one, create it in the same way as the footers below
\fancyhead[C]{Tema 2}
\fancyhead[R]{Carlos de la Torre}
\fancyfoot[R]{Página \thepage\ de \pageref{UltimaPagina}} % Empty left footer
\fancyfoot[C]{} % Empty center footer
\fancyfoot[L]{\hyperlink{Elindice}{Inicio}} % Page numbering for right footer
\renewcommand{\headrule}{\hbox to\headwidth{\color{red}\leaders\hrule height \headrulewidth\hfill}} % para poner la linea de la parte superior en rojo
\renewcommand{\sectionmark}[1]{\markboth{#1}{}} % para poder poner solo el nombre de la sección que se muestra en la pagina sin el numero
\renewcommand{\headrulewidth}{2.0pt} % Remove header underlines
\renewcommand{\footrulewidth}{0.8pt} % Remove footer underlines
\setlength{\headheight}{13.6pt} % Customize the height of the header

\numberwithin{equation}{section} % Number equations within sections (i.e. 1.1, 1.2, 2.1, 2.2 instead of 1, 2, 3, 4)
\numberwithin{figure}{section} % Number figures within sections (i.e. 1.1, 1.2, 2.1, 2.2 instead of 1, 2, 3, 4)
\numberwithin{table}{section} % Number tables within sections (i.e. 1.1, 1.2, 2.1, 2.2 instead of 1, 2, 3, 4)

\setlength\parindent{0pt} % Removes all indentation from paragraphs - comment this line for an assignment with lots of text

\newcommand{\horrule}[1]{\rule{\linewidth}{#1}} % Create horizontal rule command with 1 argument of height
\newcommand{\win}{Windows\textsuperscript{\textregistered} } % este comando sirve para poner el logotipo de la r en windows
\newcommand{\ubuntu}{Ubuntu\textsuperscript{\textregistered} } % este comando sirve para poner el logotipo de la r en windows

%----------------------------------------------------------------------------------------
%	TÍTULO Y DATOS DEL ALUMNO
%----------------------------------------------------------------------------------------

\titlehead{}%aquí se pone una foto o imagen para la portada
\title{	
\normalfont \normalsize 
\textsc{{Procesamiento Digital de Señales PDS} \\ Grado en Ingeniería Informática \\ Universidad de Granada} \\ [25pt] % Your university, school and/or department name(s)
\horrule{0.5pt} \\[0.4cm] % Thin top horizontal rule
\huge Como se realiza la compresión en MP3 % The assignment title
\horrule{2pt} \\[0.5cm] % Thick bottom horizontal rule
}
\author{Carlos de la Torre}
\date{\normalsize\today} % Incluye la fecha actual

%----------------------------------------------------------------------------------------
% DOCUMENTO
%----------------------------------------------------------------------------------------

\begin{document}

\maketitle % Muestra el Título
\begin{figure}[H]
	\centering
	%\includegraphics[width=\textwidth]{odoo_logo_rgb.png}
	\label{fig:LogoOdoo}
\end{figure}

\thispagestyle{empty}

\newpage %inserta un salto de página
\thispagestyle{empty} 

\hypertarget{Elindice}{} % es la referencia de el indice para volver a el cualquier momento
\tableofcontents % para generar el índice de contenidos

%\listoffigures

%\listoftables
\thispagestyle{empty} % Permite no mostrar el número de la página en la que se usa.
\newpage
\setcounter{page}{1} % Resetea el número de página y comienza a contar desde el número que le pases, ejemplo: \setcounter{page}{1}

%--------------------------------------------------------------------------------------------
%                                                   RESUMEN
%--------------------------------------------------------------------------------------------
\section[Resumen/Abstract]{Resumen/Abstract}
\begin{abstract}
	Estos son los puntos que hay que tratar en la memoria de la aplicación android
	\begin{itemize}
		\item Hay que poner el objetivo de la aplicación (breve descripción)
		\item Motivación de la misma (el por que he escogido hacer eso)
		\item Cuales son las consultas que he hecho con los interesados (Posibles clientes)
		\item Cuales son las consultas realizadas con los proveedores (Posibles proveedores)
		\item A nivel de programación cuales han sido las decisiones tomadas y por que
		\item 
	\end{itemize}
	\begin{multicols}{2}
		
	\end{multicols}
\end{abstract}
\thispagestyle{empty} % Permite no mostrar el número de la página en la que se usa.
\clearpage

%--------------------------------------------------------------------------------------------
%                                              Primera Sección
%--------------------------------------------------------------------------------------------
\section{Descripción del entorno de trabajo}
Para poder explicar de la mejor manera posible como poder crear un CPD de bajo coste y que se pueda utilizar en un entorno de producción sin que afecte al rendimiento se ha optado por crear un laboratorio que se asemeje a un CPD real.\\
Para ello se han utilizado 6 maquinas virtuales y una maquina física, que se encargaran de los diferentes servicios que tiene que alojar un CPD.

\begin{enumerate}
	\setlength\itemsep{1px} % Espacio vertical para la lista
	\item FIREWALL
	\item DNS
	\item DHCP
	\item LOAD BALANCED
	\item WWW
	\item FTP
	\item EMAIL
\end{enumerate}

\begin{wrapfigure}{i}{0.1\textwidth}
	\vspace{-25pt}
	\begin{center}
		%\includegraphics[width=1\textwidth]{Fabien}
	\end{center}
	\vspace{-25pt}
\end{wrapfigure}

Por supuesto que al ser un entorno virtual, y siendo solamente un laboratorio, hay algunas restricciones inherentes a dicho entorno, como por ejemplo, que desde el exterior de la red virtual solo se permite el acceso a solo un servidor web, que en este caso sera el balanceador de carga.\\
Otro aspecto a tener en cuenta en la configuración del laboratorio es que aunque se ha diseñado el mismo para tener una configuración de doble DMZ la parte de la red empresarial no se ha implementado en dicho laboratorio, ya que esta su implementación es trivial, y solo se deseaba simular la configuración de los servidores y el acceso de los mismo a Internet.\\

\subsection{Descripción de la maquina física}
Para mondar dicho laboratorio se ha utilizado una maquina que denominaremos host con la siguiente configuración, tanto de hardware como de software:
\subsubsection{Hardware}
\begin{enumerate}
	\setlength\itemsep{1px} % Espacio vertical para la lista
	\item Portátil Phoenix
	\item Placa Base: Pegatron Corporation model: H36Y
	\item Procesador: Intel\textregistered~Core\texttrademark~i5 CPU M430  @ 2.27GHz 4 nucleos x64\_86
	\item Memoria: Transcend 2 x JM1066KSN-4Gb 1066MHz (0.9ns de acceso)
	\item Gráfica: Intel\textregistered~Corporation VGA Compatible
	\item Disco Duro (OS): Samsung SSD 840 250Gb
	\item Disco Duro (VM): Seagate ST1000LM024 HN-M 1TB
	\item Red: Qualcomm Atheros AR8131 Gigabit Ethernet
	\item Wifi: Qualcomm Atheros  AR9285 Wireless Network Adapter
\end{enumerate}

\subsubsection{Software}
\begin{enumerate}
	\setlength\itemsep{1px} % Espacio vertical para la lista
	\item Sistema Operativo: Fedora Spin 21 x64\_86
	\item Kernel: 3.18.9-100.fc20.x86\_64
	\item Gestor Gráfico: Qt 4.8.6, KDE: 4.14.6
	\item HyperVisor: VMware Workstation 11.0.0 build-2305329
\end{enumerate}

\subsection{Descripción de las maquinas virtuales}
Como las maquinas virtuales utilizas parten de una misma base, describiremos la base desde la cual se han clonado todas y a continuación iremos agregando o quitando propiedades a las diferentes maquinas para que se amolden a las necesidades de los servicios que tienen que prestar dentro del CPD.\\

\subsubsection{Maquina Virtual Base}
Datos de la maquina base:\\
\begin{tabular}{|l|l|l|c|}
	\hline \multicolumn{2}{|l|}{\textbf{Hardware}} & \multicolumn{2}{l|}{\textbf{Software}} \\ 
	\hline \textbf{Procesador:} & 1 x Intel\textregistered~Core\texttrademark~i5 @ 2.27GHz & \textbf{Sistema Operativo:} & Centos 7 x64\_86 \\ 
	\hline \textbf{Memoria:} & 384 Mb & \textbf{Kernel:} &  \\ 
	\hline \textbf{Disco Duro:} & 8Gb SATA & \textbf{Gestor Gráfico:} & Sin gestor \\ 
	\hline \textbf{Red 1:} & VmNet8 NAT  & \multicolumn{2}{c|}{\textbf{Addons:}} \\ 
	\hline \textbf{Red 2:} & VmNet1 Host Only & \multicolumn{2}{c|}{Webmin, Apache, MySQL, Issue panel} \\ 
	\hline 
\end{tabular} 

\subsubsection{Maquina DNS}
En esta maquina como se utilizan los servicios de DNS, DHCP y ninguno de los dos son servicios que necesiten demasiados recursos lo que se ha hecho es bajar la configuración de la memoria hasta los 256 Mb y solo una tarjeta de red, y por supuesto se ha añadido el software necesario para poder configurar ambos servicios, aparte se han añadido también los módulos necesarios de WebMin.\\
\begin{tabular}{|l|l|l|c|}
	\hline \multicolumn{2}{|l|}{\textbf{Hardware}} & \multicolumn{2}{l|}{\textbf{Software}} \\ 
	\hline \textbf{Procesador:} & 1 x Intel\textregistered~Core\texttrademark~i5 @ 2.27GHz & \textbf{Sistema Operativo:} & Centos 7 x64\_86 \\ 
	\hline \textbf{Memoria:} & 256 Mb & \textbf{Kernel:} &  \\ 
	\hline \textbf{Disco Duro:} & 8Gb SATA & \textbf{Gestor Gráfico:} & Sin gestor \\ 
	\hline \textbf{Red 1:} & VmNet8 NAT  & \multicolumn{2}{c|}{\textbf{Addons:}} \\ 
	\hline \multicolumn{4}{|c|}{Webmin, Apache, MySQL, Issue panel, plugin DNS, DHCP, FTP} \\ 
	\hline 
\end{tabular} 

\subsection{Entorno DMZ}
Como ya se ha explicado varias veces en clase esta parte de la topología de red, denominada así por su similitud con una zona de un conflicto armado, se utiliza para asegurar aquella zona de la topología, la cual aun teniendo sus propios sistemas de seguridad, se aísla de la red de manera física aportando así un extra en dicha seguridad.\\

\begin{figure}[H]
	\centering
	\begin{subfigure}[b]{0.3\textwidth}
		%\includegraphics[width=\textwidth]{3-openerp-wizard.jpg}
		\caption{Bienvenida del asistente}
		\label{fig:Bienvenida}
	\end{subfigure}
	~
	\begin{subfigure}[b]{0.3\textwidth}
		%\includegraphics[width=\textwidth]{4-openerp-wizard.jpg}
		\caption{Aceptamos la licencia}
		\label{fig:Licencia}
	\end{subfigure}
	~
	\begin{subfigure}[b]{0.3\textwidth}
		%\includegraphics[width=\textwidth]{5-openerp-wizard.jpg}
		\caption{Elegimos que instalar}
		\label{fig:Elección}
	\end{subfigure}
	~
	\begin{subfigure}[b]{0.3\textwidth}
		%\includegraphics[width=\textwidth]{6-openerp-wizard.jpg}
		\caption{Componentes a instalar}
		\label{fig:Componentes}
	\end{subfigure}
	~
	\begin{subfigure}[b]{0.3\textwidth}
		%\includegraphics[width=\textwidth]{7-openerp-wizard.jpg}
		\caption{Datos de conexión BD}
		\label{fig:DBConn}
	\end{subfigure}
	~
	\begin{subfigure}[b]{0.3\textwidth}
		%\includegraphics[width=\textwidth]{8-openerp-wizard.jpg}
		\caption{Ruta de instalación}
		\label{fig:Ruta}
	\end{subfigure}
	~
	\begin{subfigure}[b]{0.3\textwidth}
		%\includegraphics[width=\textwidth]{9-openerp-wizard.jpg}
		\caption{Proceso de instalación}
		\label{fig:Instalación}
	\end{subfigure}
	~
	\begin{subfigure}[b]{0.3\textwidth}
		%\includegraphics[width=\textwidth]{10-openerp-wizard.jpg}
		\caption{Final de la instalación}
		\label{fig:Final}
	\end{subfigure}
	\caption{Instalación en \win}
	\label{fig:InstalaciónWin}
\end{figure}

\subsection{Topología completa de Red}
\begin{figure}[H]
		%\includegraphics[width=\textwidth]{}
		\caption{Topología de Red}
		\label{fig:bienvenida}
\end{figure}


%--------------------------------------------------------------------------------------------
%                                              Segunda Sección
%--------------------------------------------------------------------------------------------
\section{Descripción de la parte del Firewall}

\subsection{Posible Software para el uso}

\subsection{Posibilidades de Hardware para el balanceador}

\subsection{Ventajas y desventajas, opiniones, conclusiones, configuraciones}

%\lstinputlisting{httpd.cfg}

%--------------------------------------------------------------------------------------------
%                                              Tercera Sección
%--------------------------------------------------------------------------------------------
\section{Descripción de la parte del Balanceador de Carga}

\subsection{Posible Software para el uso}

\subsection{Posibilidades de Hardware para el balanceador}

\subsection{Ventajas y desventajas, opiniones, conclusiones, configuraciones}
\begin{itemize}
	\item Procesador: AMD Opteron(tm) Processor 4386 , 1 core
	\item Memoria: 1024 MB Registered
	\item HDD: 10 GB Virtual HD
\end{itemize}

%\lstinputlisting{ssl.cfg}

%--------------------------------------------------------------------------------------------
%                                              Cuarta Sección
%--------------------------------------------------------------------------------------------
\section{Descripción de la parte del Servidor DNS}

\subsection{Posible Software para el uso}

\subsection{Posibilidades de Hardware para el balanceador}

\subsection{Ventajas y desventajas, opiniones, conclusiones, configuraciones}

%--------------------------------------------------------------------------------------------
%                                              Quinta Sección
%--------------------------------------------------------------------------------------------
\section{Descripción de la parte del Servidor Apache}

\subsection{Posible Software para el uso}

\subsection{Ventajas y desventajas, opiniones, conclusiones, configuraciones}

%--------------------------------------------------------------------------------------------
%                                              Sexta Sección
%--------------------------------------------------------------------------------------------
\section{Descripción de la parte del Servidores Base de Datos}

\subsection{Posible Software para el uso}

\subsection{Ventajas y desventajas, opiniones, conclusiones, configuraciones}
%--------------------------------------------------------------------------------------------
%                                              Conclusiones
%--------------------------------------------------------------------------------------------
\section{Conclusiones}

%--------------------------------------------------------------------------------------------
%                           Bibliografia y agradecimientos
%--------------------------------------------------------------------------------------------
\bibliography{citas} %archivo citas.bib que contiene las entradas 
\bibliographystyle{plain} % hay varias formas de citar
\label{UltimaPagina} % esto es para marcar que es la ultima pagina

\end{document}