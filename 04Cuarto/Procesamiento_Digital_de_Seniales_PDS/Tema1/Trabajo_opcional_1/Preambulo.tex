%%%%%%%%%%%%%%%%%%%%%
% Short Sectioned Assignment LaTeX Template Version 1.0 (5/5/12)
% This template has been downloaded from: http://www.LaTeXTemplates.com
% Original author:  Frits Wenneker (http://www.howtotex.com)
% License: CC BY-NC-SA 3.0 (http://creativecommons.org/licenses/by-nc-sa/3.0/)
%%%%%%%%%%%%%%%%%%%%%

%----------------------------------------------------------------------------------------
%	PACKAGES AND OTHER DOCUMENT CONFIGURATIONS
%----------------------------------------------------------------------------------------

\documentclass[paper=a4, fontsize=10pt]{scrartcl} % A4 paper and 11pt font size

% ---- Margenes y Tamaño del Folio -----
\usepackage{geometry}
\geometry{a4paper, total={210mm,297mm}, left=25mm, right=25mm, top=20mm, bottom=25mm,}

% ---- Entrada y salida de texto -----
\usepackage[T1]{fontenc} % Use 8-bit encoding that has 256 glyphs
\usepackage[utf8]{inputenc}
%\usepackage{fourier} % Use the Adobe Utopia font for the document - comment this line to return to the LaTeX default


% ---- Idioma y Figuras --------
% Selecciona el español para palabras introducidas automáticamente, p.ej. "septiembre" en la fecha
% y especifica que se use la palabra Tabla en vez de Cuadro
\usepackage[spanish, es-tabla]{babel} 
\usepackage{wrapfig} % es para que las imagenes puedan estar entre el texto
\usepackage{subcaption} % para las imagenes en columnas
%\usepackage{svg} % paquete para poder insertar graficos en formato svg y que se impriman en pdf

% ---- Otros paquetes ----
\usepackage{amsmath,amsfonts,amsthm} % Math packages
\usepackage{graphics,graphicx,floatrow} %para incluir imágenes y notas en las imágenes
\usepackage{lscape, pdflscape} %Permite poner paginas "tumbadas" en un documento con las paginas "rectas".
\usepackage{pgfgantt} % Para realizar los diagramas de Gantt
\usepackage{fancyhdr, extramarks} % Custom headers and footers

% ---------- Para hacer tablas comlejas ---------
\usepackage{multirow,threeparttable}
\usepackage{multicol} % Required for creating multiple columns in slides
\usepackage{sectsty} % Allows customizing section commands
\usepackage{tabularx} % Tables with width auto ajust
%\usepackage{tablefootnote} % para poner pie de notas en las tablas solo en tabular envirotment

%------------------------------------------------
%                  Links y Moneda
\usepackage[hidelinks]{hyperref} % para los enlaces de la bibliografía
\hypersetup{
	colorlinks,
	citecolor=blue,
	filecolor=brown,
	linkcolor=black,
	urlcolor=blue
}
\usepackage{eurosym} % para el euro

%------------------------------------------------
% Colors
%\usepackage[usenames,table]{xcolor}	 % Required for custom colors only with tikz package
\usepackage{xcolor, colortbl} % color en las tablas
% Define a few colors for making text stand out within the presentation
\definecolor{mygreen}{RGB}{44,85,17}
\definecolor{myblue}{RGB}{34,31,217}
\definecolor{mybrown}{RGB}{194,164,113}
\definecolor{myred}{RGB}{255,66,56}
\definecolor{mygrey}{RGB}{230,230,230}
% Use these colors within the presentation by enclosing text in the commands below
\newcommand*{\mygreen}[1]{\textcolor{mygreen}{#1}}
\newcommand*{\myblue}[1]{\textcolor{myblue}{#1}}
\newcommand*{\mybrown}[1]{\textcolor{mybrown}{#1}}
\newcommand*{\myred}[1]{\textcolor{myred}{#1}}
\newcommand*{\mygrey}[1]{\textcolor{mygrey}{#1}}

%------------------------------------------------
%             Listas
\usepackage{listings} % para los listados de código
\usepackage{tabto} % para utilizar los tab en las listas
\lstdefinestyle{C++}{
	belowcaptionskip=1\baselineskip,
	frame=L,
	xleftmargin=\parindent,
	showstringspaces=false,
	basicstyle=\footnotesize\ttfamily,
	keywordstyle=\bfseries\color{green!40!black},
	%commentstyle=\itshape\color{purple!40!black},
	commentstyle=\color{gray}\upshape,
	identifierstyle=\color{blue},
	stringstyle=\color{orange},
	%tagstyle=\color{darkblue}\bf,
	language=C++,
	extendedchars=true, 
	breaklines=true,
	breakatwhitespace=true,
	emph={},
	emphstyle=\color{red},
	columns=fullflexible,
	morestring=[b]",
	morecomment=[s]{/**}{*/},
	morecomment=[s][\color{forestgreen}]{<!--}{-->},
	otherkeywords={},
	morekeywords={IP,num,string,Servicio}
}
\lstdefinestyle{java}{
	language=Java,
	aboveskip=3mm,
	belowskip=3mm,
	showstringspaces=false,
	columns=flexible,
	basicstyle={\footnotesize\ttfamily},
	numberstyle={\tiny},
	numbers=left,
	keywordstyle=\color{blue},
	commentstyle=\color{dkgreen},
	stringstyle=\color{mauve},
	breaklines=true,
	breakatwhitespace=true,
	tabsize=3
}

\lstdefinestyle{bytecode}{
	otherkeywords={invokedynamic},
	language=JVMIS,
	aboveskip=3mm,
	belowskip=3mm,
	showstringspaces=false,
	columns=flexible,
	basicstyle={\footnotesize\ttfamily},
	numberstyle={\tiny},
	numbers=left,
	keywordstyle=\color{blue},
	commentstyle=\color{dkgreen},
	stringstyle=\color{mauve},
	breaklines=true,
	breakatwhitespace=true,
	tabsize=3
}

\lstdefinestyle{fsharp} {	
	otherkeywords={ let!, return!, do!, yield!, use!, var, from, select, where, order},
	keywordstyle=\color{blue},
	sensitive=true,
	aboveskip=3mm,
	belowskip=3mm,
	showstringspaces=false, 
	columns=flexible,
	basicstyle={\footnotesize\ttfamily},
	numberstyle={\tiny},
	numbers=left,
	breaklines=true,
	upquote=true,
	tabsize=3,
	morecomment=[l][\color{dkgreen}]{///},
	morecomment=[l][\color{dkgreen}]{//},
	morecomment=[s][\color{dkgreen}]{{(*}{*)}},
	morestring=[b]",
	showstringspaces=false,
	literate={`}{\`}1,
	stringstyle=\color{mauve},
	morekeywords={let, new, match, with, rec, 
		open, module, namespace, type, of, member, 
		and, for, while, true, false, in, do, begin, 
		end, fun, function, return, yield, try, val, 
		mutable, if, then, else, cloud, async, static, 
		use, abstract, interface, inherit, finally, maybe, option }
}

\lstdefinestyle{csharp} {
	aboveskip=3mm,
	belowskip=3mm,
	showstringspaces=false,
	columns=flexible,
	basicstyle={\footnotesize\ttfamily},
	numberstyle={\tiny},
	numbers=left,
	keywordstyle=\color{blue},
	commentstyle=\color{dkgreen},
	stringstyle=\color{mauve},
	breaklines=true,
	breakatwhitespace=true,
	tabsize=3,
	morecomment = [l]{//}, 
	morecomment = [l]{///},
	morecomment = [s]{/*}{*/},
	morestring=[b]", 
	sensitive = true,
	morekeywords = {async, await, abstract,  
		event,  new,  struct,
		as,  explicit,  null,  switch,
		base,  extern,  object,  this,
		bool,  false,  operator,  throw,
		break,  finally,  out,  true,
		byte,  fixed,  override,  try,
		case,  float,  params,  typeof,
		catch,  for,  private,  uint,
		char,  foreach,  protected,  ulong,
		checked,  goto,  public,  unchecked,
		class,  if,  readonly,  unsafe,
		const,  implicit,  ref,  ushort,
		continue,  in,  return,  using,
		decimal,  int,  sbyte,  virtual,
		default,  interface,  sealed,  volatile,
		delegate,  internal,  short,  void,
		do,  is,  sizeof,  while,
		double,  lock,  stackalloc,   
		else,  long,  static,   
		enum,  namespace,  string }
}

\lstdefinestyle{scala} {  
	morekeywords={ abstract,case,catch,
		char,class,
		def,else,extends,final,
		if,import,
		match,module,new,null,object,
		override,package,private,protected,
		public,return,super,this,throw,
		trait,try,type,val,var,with,implicit,
		macro,sealed
	},
	sensitive,
	morecomment=[l]//,
	morecomment=[s]{/*}{*/},
	morestring=[b]",
	morestring=[b]',
	aboveskip=3mm,
	belowskip=3mm,
	showstringspaces=false,
	columns=flexible,
	basicstyle={\footnotesize\ttfamily},
	numberstyle={\tiny},
	numbers=left,
	keywordstyle=\color{blue},
	commentstyle=\color{dkgreen},
	stringstyle=\color{mauve},
	breaklines=true,
	breakatwhitespace=true,
	tabsize=3
}
\lstset{ %
	literate={á}{{\'a}}1 {ó}{{\'o}}1 {ñ}{{\~n}}1 {é}{{\'e}}1 {í}{{\'i}}1 {ú}{{\'u}}1, % esto es para los acentos en utf8
	backgroundcolor=\color{mygrey},     % elegir el color de fondo; debe agregar \usepackage{color} o \usepackage{xcolor}
	basicstyle=\footnotesize\ttfamily,                 % el tamaño de las fuentes que se utilizan para el código
	breakatwhitespace=false,                % establece si se rompe automáticos sólo debe ocurrir en los espacios en blanco
	breaklines=true,                               % establece los saltos de línea automática
	captionpos=b,                                  % establece el título de la posición hacia abajo
	commentstyle=\color{magenta},         % estilo de comentarios
	deletekeywords={...},                      % si desea eliminar palabras clave de la lengua dada
	escapeinside={\%*}{*)},                % si desea agregar LaTeX dentro de su código
	extendedchars=true,                        % le permite utilizar caracteres no ASCII; de 8-bits de codificaciones sólo, no funciona con UTF-8
	frame=single,                                   % agrega un marco alrededor del código
	keepspaces=true,                            % mantiene espacios de texto, útiles para mantener la sangría de código (posiblemente necesita columnas = flexible)
	keywordstyle=\color{blue},              % estilo palabra clave
	%language=bash,                            % el lenguaje del código
	morekeywords={*,...},                     % si quieres añadir más palabras clave para el conjunto
	numbers=left,                                   % dónde poner la línea de números; los valores posibles son (ninguno, izquierda, derecha)
	numbersep=5pt,                               % hasta qué punto las alineaciones números son a partir del código
	numberstyle=\footnotesize\color{blue},        % el estilo que se utiliza para los números de línea-
	rulecolor=\color{black},                    % si no se establece, el marco-color puede ser cambiado en saltos de línea en el texto no-negro (por ejemplo, los comentarios (verde aquí))
	showspaces=false,                            % mostrar espacios añadiendo todas partes subrayado particulares; anula 'showstringspaces'
	showstringspaces=false,                   % subrayar espacios dentro de las cadenas sólo
	showtabs=false,                                % Mostrar pestañas dentro de las cadenas de subrayado particulares
	stepnumber=1,                                 % el paso entre dos números de línea. Si es 1, se numerará cada línea
	stringstyle=\color{red},              % cadena estilo literal
	tabsize=2,                                         % establece TABSIZE predeterminado a 2 espacios
	title=\lstname                                   % mostrar el nombre de archivo de los archivos incluidos con \ lstinputlisting; también tratar de leyenda en vez de título
}
%------------------------------------------------

\pagestyle{fancyplain} % Makes all pages in the document conform to the custom headers and footers
\fancyhead[LO]{Solución a Problemas de Abstracción} % No page header - if you want one, create it in the same way as the footers below
\fancyhead[C]{Tema 2}
\fancyhead[R]{Carlos de la Torre}
\fancyfoot[R]{Página \thepage\ de \pageref{UltimaPagina}} % Empty left footer
\fancyfoot[C]{} % Empty center footer
\fancyfoot[L]{\hyperlink{Elindice}{Inicio}} % Page numbering for right footer
\renewcommand{\headrule}{\hbox to\headwidth{\color{red}\leaders\hrule height \headrulewidth\hfill}} % para poner la linea de la parte superior en rojo
\renewcommand{\sectionmark}[1]{\markboth{#1}{}} % para poder poner solo el nombre de la sección que se muestra en la pagina sin el numero
\renewcommand{\headrulewidth}{2.0pt} % Remove header underlines
\renewcommand{\footrulewidth}{0.8pt} % Remove footer underlines
\setlength{\headheight}{13.6pt} % Customize the height of the header

\numberwithin{equation}{section} % Number equations within sections (i.e. 1.1, 1.2, 2.1, 2.2 instead of 1, 2, 3, 4)
\numberwithin{figure}{section} % Number figures within sections (i.e. 1.1, 1.2, 2.1, 2.2 instead of 1, 2, 3, 4)
\numberwithin{table}{section} % Number tables within sections (i.e. 1.1, 1.2, 2.1, 2.2 instead of 1, 2, 3, 4)

\setlength\parindent{0pt} % Removes all indentation from paragraphs - comment this line for an assignment with lots of text

\newcommand{\horrule}[1]{\rule{\linewidth}{#1}} % Create horizontal rule command with 1 argument of height
\newcommand{\win}{Windows\textsuperscript{\textregistered} } % este comando sirve para poner el logotipo de la r en windows
\newcommand{\ubuntu}{Ubuntu\textsuperscript{\textregistered} } % este comando sirve para poner el logotipo de la r en windows